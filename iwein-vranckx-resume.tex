% !TeX document-id = {29e3355d-f5f2-4d79-860a-225b1558ea09}
% !TeX spellcheck = en_GB
% !TeX program = xelatex 
%%%%-shell-escape

% LaTeX source of my resume
% =========================

% Commented for easy reuse... ;)

% See the `README.md` file for more info.

% This file is licensed under the CC-NC-ND Creative Commons license.


% Start a document with the here given default font size and paper size.
\documentclass[10pt,a4paper]{article}

\usepackage{fontawesome}
\usepackage{fontspec}
% Set the page margins.
\usepackage[a4paper,margin=0.75in]{geometry}
\usepackage{bibentry}
\usepackage[a-1b]{pdfx} 

\usepackage{draftwatermark}
\SetWatermarkText{DRAFT}
\SetWatermarkScale{1}
\SetWatermarkColor[gray]{1}

% Setup the language.
%\usepackage[english]{babel}
%\hyphenation{Some-long-word}

\usepackage{xcolor}
\newcommand{\deemph}[1]{{\color{black!40}#1}}

% fancy quotes
\definecolor{quotemark}{gray}{0.6}
\makeatletter
\def\fquote{%
	\@ifnextchar[{\fquote@i}{\fquote@i[]}%]
}%
\def\fquote@i[#1]{%
	\def\tempa{#1}%
	\@ifnextchar[{\fquote@ii}{\fquote@ii[]}%]
}%
\def\fquote@ii[#1]{%
	\def\tempb{#1}%
	\@ifnextchar[{\fquote@iii}{\fquote@iii[]}%]
}%
\def\fquote@iii[#1]{%
	\def\tempc{#1}%
	\vspace{1em}%
	\noindent%
	\begin{list}{}{%
			\setlength{\leftmargin}{0.1\textwidth}%
			\setlength{\rightmargin}{0.1\textwidth}%
		}%
		\item[]%
		\begin{picture}(0,0)%
			\put(-15,-5){\makebox(0,0){\scalebox{4}{\textcolor{quotemark}{``}}}}%
		\end{picture}%
		\begingroup\itshape}%
	%%%%********************************************************************
	\def\endfquote{%
		\endgroup\par%
		\makebox[0pt][l]{%
			\hspace{0.8\textwidth}%
			\begin{picture}(0,0)(0,0)%
				\put(15,15){\makebox(0,0){%
						\scalebox{4}{\color{quotemark}''}}}%
		\end{picture}}%
		\ifx\tempa\empty%
		\else%
		\ifx\tempc\empty%
		\hfill\rule{100pt}{0.5pt}\\\mbox{}\hfill\tempa,\ \emph{\tempb}%
		\else%
		\hfill\rule{100pt}{0.5pt}\\\mbox{}\hfill\tempa,\ \emph{\tempb},\ \tempc%
		\fi\fi\par%
		\vspace{0.5em}%
	\end{list}%
}%
\makeatother
%%%%***********************************

% Makes resume-specific commands available.
\usepackage{resume}

\usepackage{academicons}
\definecolor{orcidlogocol}{HTML}{A6CE39}

\begin{document}  % begin the content of the document
\sloppy  % this to relax whitespacing in favour of straight margins

%\nobibliography{ref}
%\bibliographystyle{unsrt}

% title on top of the document
\maintitle{Iwein Vranckx}{October 14, 1978}{Last update on \today}
\nobreakvspace{0.3em}  % add some page break averse vertical spacing

% \noindent prevents paragraph's first lines from indenting
% \mbox is used to obfuscate the email address
% \sbull is a spaced bullet
% \href well..
% \\ breaks the line into a new paragraph
%\noindent\href{mailto:iwein.vranckx@kuleuven.be}{iwein.vranckx\mbox{}@\mbox{}kuleuven.be}\sbull
\noindent\href{mailto:ivranckx@gmail.com}{ivranckx\mbox{}@\mbox{}gmail.com}\sbull
016.43.83.29\sbull 0475.75.34.92\sbull
\href{https://github.com/ivranckx}{Github}\sbull \href{https://www.researchgate.net/profile/Iwein_Vranckx}{Researchgate}\sbull \href{http://linkedin.com/in/ivranckx}{Linkedin} \\
\href{https://www.google.com/maps/place/3202+Rillaar}{Oudenbos  81, 
3202\thinspace Rillaar, Belgium}\sbull%{\large \sc ld}\sbull
\href{https://orcid.org/0000-0002-2439-6906}{\textcolor{orcidlogocol}{\aiOrcid} \hspace{0.2mm} 0000-0002-2439-6906}
\spacedhrule{0.9em}{-0.4em}  % a horizontal line with some vertical spacing before and after
\begin{fquote}[J. Conrad][Heart of Darkness][1902]
	\large	Outliers are like thinking about an enigma. There it is before you - smiling, frowning, inviting, grand, mean, insipid, or savage, and always mute with an air of whispering, ‘Come and find out'.
\end{fquote}
%\spacedhrule{0.9em}{-0.4em}  % a horizontal line with some vertical spacing before and after
%\roottitle{}  % a root section title
%\vspace{-1.3em}  % some vertical spacing
%\begin{multicols}{2}  % open a multicolumn environment
%\noindent \emph{Entrepeneurial geek with roots in the open source movement. Passionately enabling software-related teams to deliver. Creates/\,structures/\,improves processes. Architects and helps to implement future-proof solutions. Coaches both individuals and groups.}
%%\\
%%At the age of seven (1989) Cies wrote his first lines of code in a \acr{LOGO}-like language on an \acr{MSX} (pre-\acr{PC}).  Two years later he attended a conference on an emerging new technology, the Internet, at the Erasmus University from which he would graduate 16 years later (2000) with a degree in Business Computer Science.
%%After being introduced to the open source movement in 1997, he taught himself a variety of skills including system administration and programming (Bash, Python, Ruby \& \CPP).  By 2002 he got his pet project \acr{KT}urtle ---a zero-entry-barrier programming environment--- included in \acr{KDE}'s \emph{edu} module, and thereby in almost every Linux distribution.
%%Extensively travelled Europe and Asia after graduating. On return his professional life became a mix of hands-on work in startups and web software agencies, while settling as a husband and father in personal life.
%\end{multicols}
%\spacedhrule{0em}{-0.4em}
\vspace{-4.3em}  % some vertical spacing
\roottitle{Experience}

\headedsection
  {\href{https://www.tomra.com/en/}{\acr{TOMRA Sorting NV}}}
  {\textsc{Haasrode, Belgium}} { \\ %
%  \vspace{+0.2em} 
%  \deemph{Maximization of ML and AI usage through the developing of new, improved classifiers. }
%  \vspace{+0.2em} 
  \headedsubsection
  {Senior Research Scientist - PhD Student}
  {Oct 2016 -- Nov 2020}
  {\bodytext{
	%Robust statistics to provide statistical methods which are not unduly affected
	%by outliers: multivariate observations that deviate from normal samples. 
	Classical statistical methods (e.g. SVM's, PCA, QDA, CNN's, ...) often have poor performance on datasets with outliers. Although it is fairly
	easy to show the dramatic effects caused by just
	one outlier, the use of robust statistics in industrial machine
	learning tasks is only gradually being recognized in the last years.  TOMRA Sorting develops industrial food inspection machines which process several
	gigabytes of high dimensional measurements in milliseconds, thereby pushing against the boundaries of the available computational power. 
	%Furthermore, the amount of	data is expected to grow over the coming years.
	My research activities already proved that the cross-fertilization of industrial machine classifiers and robust statistical methods was beneficial. However, the extreme high computational
	load continuously forces research into massive parallelization technologies and – more importantly - the research and development of new, robust and sparse machine learning 
	methods which can process large datasets under real-time constraints.\\	
	To this end I initialised an intra disciplinary project between KULeuven, TOMRA and VLAIO which was directed towards strategic classification research with a clear defined economic finality. 	
	I was based at the MeBioS (Mechatronics, Biostatistics and Sensors) division, under the supervision of B. De Ketelaere, M. Hubert and P. J. Rousseeuw. \vspace{0.2em} \\	
	\deemph{Keywords: Python/MATLAB, C++, Git, robust (non)linear regression and classification methods.}
}}

  \vspace{+1em}  
  \headedsubsection
    {Senior Research Scientist}
    {2014 -- 2016}
    {\bodytext{
 GP-GPU's are powerful tools that are well-suited to unravelling complex, real-world problems, and form the foundation of today's deep learning frameworks such as TensorFlow \& GPU accelerated optimization of SVM's. 
Due to the time criticality of our industrial classification tasks in day-to-day operations, combined with the vast amount of spectroscopic data, research into new, high performance (non-linear) modelling was essential. 
My research therefore focused specifically on the computational improvement of (ensembles of) LS-SVM's based on GPGPU's, which where then applied to spectroscopic datasets of various food related products. We effectively demonstrated the improved machine efficiency and computation times under real sorting conditions. \\
	\deemph{Keywords: MATLAB, C++, cuBLAS, CUDA, Git, global optimization, LS-SVM's, loss functions.}
}}

 \vspace{+1em}  
  \headedsubsection
{Research Scientist, digital signal processing}
{2010 -- 2014}
{\bodytext{
		%Data mining, exploratief onderzoek en implementatie van chemometrischemodellen,pre-processing.Mijn initieel werk was het onderzoek naar de werking en haalbaarheid van latente regressie methoden (PCR, O-PLS, PLS-DA, ... ). Prototypes werdendoor mij geïmplementeerd in Java. Daarnaast had ik een groot aandeel in de inverse en-gineering van bestaande classificatoren.2. 
		
		\textit {Smart Sort} is TOMRA's commercial name for an algorithm that I developed in two years. It enables machines to auto-configure itself: after providing training data the framework automatically ensures the most efficient machine settings for a given specificity-sensitivity trade-off setting. Although multi-class classification is entirely  supported through ensemble learning, feedback from customers surprisingly learned us that it is mostly used as (an ensemble of) one class classifier(s). The algorithm is trained on good product(s) only, in turn demanding that the machine rejects all unknown, foreign materials such as glass, plastics and stones. It is currently still sold as a 6k€ machine option, and it became the de-facto classifier standard for tobacco sorting as it reduces machine change-over time dramatically. I implemented the first prototypes entirely in Java using test-driven development, and guided the software team towards it commercial/persistent implementation afterwards.
		
		
	\deemph{Keywords: Java, Octave, Subversion, optimization, regression, KDE, ensembles (boosting). }}
}}

\headedsection
{\href{https://www.unizo.be/}{\acr{Unizo HQ}}}
{\textsc{Brussels, Belgium}} {%
  \vspace{+0.2em} 
	\headedsubsection
	{ICT-Consultant}
	{2010 -- 2010}
	{\bodytext{
	I managed ICT change processes for SME's related to the integration of ERP, CRM, webshop, .. software. I also took care of the daily incoming questions, as well as the organization of training courses, workshops and networking events.
	}}}

\headedsection
{\href{https://www.sintmartinusscholen.be/}{\acr{Sint-Martinuscollege Overijse, campus Lipsius}}}
{\textsc{Overijse, Belgium}} {%
	\vspace{+0.2em} 
	\headedsubsection
	{Science and mathematics teacher}
	{2009 -- 2010}
	{\bodytext{Par-time biology, maths - and science teacher for the 2' and 3' degree students.
}}}

%\vspace{+0.2em} 
%{}{\acr{Lipsius college}}
%{\textsc{Overijse, Belgium}} {%	
%	\headedsubsection
%	{Science and mathematics teacher}
%	{2010 -- 2010}
%	{\bodytext{Par-time teacher for the 2' and 3' degree students.}}}
%	}
\vspace{+1em} 
%\begin{center}
%  \emph{\small Please refer to \href{http://www.linkedin.com/in/ivranckx}{my LinkedIn profile} for a more complete list of work experience.}
%\end{center}


\spacedhrule{-0.2em}{-0.4em}
%\newpage
\roottitle{Education}

\headedsection
  {\href{https://wet.kuleuven.be/english}{KU Leuven, faculty of Science}}
  {\textsc{Leuven, Belgium}} {%
  \headedsubsection
    {Doctor of Science (PhD): Statistics (dr.)}
    {2016 -- 2020}{\\Development of real-time, robust statistical methods with novel applications in food sorting}
}
\vspace{+0.2em}
\headedsection
  {\href{https://eng.kuleuven.be/en}{KU Leuven, faculty of Engineering Science}}
  {\textsc{Leuven, Belgium}} {%
  \headedsubsection
    {Master of Electrical Engineering: Information Systems and signal Processing (ir.)}
    {2005--2009} {\\Development of an SVM-based OCR for Latin-Greek manuscripts}
}

\vspace{+0.2em}
\headedsection
  {\href{https://www.uantwerpen.be/en/}{University of Antwerp, faculty of Biomedical Sciences}}
  {\textsc{Antwerp, Belgium}} {%
  \headedsubsection
    {Master of Biomedical Sciences: Neurosciences Research (lic.)}
    {2004 -- 2005} {\\Reduction of ring artefacts on \acr{$\mu$-CT} images}
}

\vspace{+0.2em}
\headedsection
{\href{https://www.kdg.be/}{Karel de Grote-Hogeschool}}
{\textsc{Antwerp, Belgium}} {%
	\headedsubsection
	{Master in Electronics and ICT Engineering Technology (ing.)}
	{2001 -- 2004} {\\Tristan: data acquisition software for a heat treatment production process}
}

\vspace{+0.2em}
\headedsection
{\href{https://www.kdg.be/}{Karel de Grote-Hogeschool}}
{\textsc{Antwerp, Belgium}} {%
	\headedsubsection
	{Bachelor of Electromechanics}
	{1998 -- 2001} {\\Development of a (SQL-based) database application for the registration of production data}
}

\spacedhrule{0.5em}{-0.4em}

\roottitle{Publications}

\inlineheadsection  % special section that has an inline header with a 'hanging' paragraph
{Published:}
{ 
	\begin{itemize}
		\item I.~Vranckx, De~Ketelaere, B., M.~Hubert, and P.~J. Rousseeuw (2020). Development of real-time, robust statistical methods with novel applications in food sorting. Dissertation-thesis. 	
	%\hangindent=3.5mm
	%\hangafter=1		
	\item I. Vranckx, J. Raymaekers, B. De Ketelaere, P. J. Rousseeuw, M. Hubert (2021).
	\newblock Real-time discriminant analysis in the presence of label and measurement noise.
	\newblock {\em Chemometrics and Intelligent Laboratory Systems\/}~{\em 208}.	
	
	\item De~Ketelaere, B., M.~Hubert, J.~Raymaekers, P.~J. Rousseeuw, and I.~Vranckx	(2020).
	\newblock Real-time outlier detection for large datasets by {RT-DetMCD}.
	\newblock {\em Chemometrics and Intelligent Laboratory Systems\/}~{\em 199}. 		
	
	\item Raymaekers, J., P.~J. Rousseeuw, and I.~Vranckx (2018).
	\newblock Discussion of ``{T}he power of monitoring: how to make the most of a contaminated multivariate sample''.
	\newblock {\em Statistical Methods \& Applications\/}~{\em 27}, 589--594. 
	\end{itemize}
}
\vspace{0.5em}
\inlineheadsection
{Under review:}
{

\begin{itemize}
	\item 	J. Schreurs, I. Vranckx, B. De Ketelaere, M. Hubert, J. A.K. Suykens, P. J. Rousseeuw (2020). Outlier detection in non-elliptical data by kernel MRCD. \textit{ArXiv e-prints}, arXiv:2008.02046. I. Vranckx and J. Schreurs contributed equally to this work.
\end{itemize}
}

\vspace{0.5em}
\inlineheadsection
{Working papers:}
{	
\begin{itemize}
	\item 	I. Vranckx, J. Schreurs. Robust and sparse support vector machines for data sets with label and measurement noise. I. Vranckx and J. Schreurs contributed equally to this work.
\end{itemize}
}

\spacedhrule{1.6em}{-0.4em}

\roottitle{Skills \& Interests (non-exhaustive)}

\inlineheadsection  % special section that has an inline header with a 'hanging' paragraph
{Software interests/expertise:}
{Recruitment of software and research engineers.  Good knowledge of agile and Test Driven Development methodologies, functional and OO-based programming, version control. I enjoy writing \nsp Python/\nsp Java/Julia/\CPP~and Latex. Well experienced with industrial algorithm research, development \& implementation. Sublime text, Visual Studio, Gitkraken and Pycharm user.}

\vspace{0.5em}
\inlineheadsection
{Natural languages:}
{Dutch \emph{(mother tongue)}, English \emph{(professional proficiency)}, French \emph{(elementary proficiency)}}
\vspace{0.5em}
\inlineheadsection
{Programming languages:}
{MATLAB, Octave, R, Python, Julia, C++, CUDA, Java, SQL, R (RobustBase), ArrayFire, BLAS/LAPACK/LINPACK, Spark.}
\vspace{0.5em}
\inlineheadsection	
{Research interests/expertise:}
{Data mining, (robust) calibration methods, (non-linear) classification, robust statistics, big data, CNN's, real-time GP-GPU \& MapReduce based processing, machine learning pattern recognition techniques, kernel embeddings of distributions, high dimensional, collinear data, Least Squares Support Vector Machines, anomaly \& fraud detection, Kaggle datasets, multivariate global optimization methods, chemometrics,  OOP-programming, HPC.}
\vspace{0.5em}
\inlineheadsection	
  {General interests:}
  {Sourdough bread baking and jogging (i.e. my new COVID related hobbies), chess, BZFlag, Far Cry, cycling, typography (e.g. \LaTeX), fraud detection, macrobiotic food, cooking, intra-disciplinary research collaboration, valorization of innovation, travel, speaking at conferences, lunch \& learns, quality time with my three kinds.}

\vspace{+2em} 
\begin{center}
	\emph{\small Please refer to \href{https://github.com/ivranckx/CV_Iwein_Vranckx/blob/main/iwein-vranckx-resume.pdf}{Github} for updates of this document.}
\end{center}
\spacedhrule{-0.2em}{-0.4em}

\end{document}

