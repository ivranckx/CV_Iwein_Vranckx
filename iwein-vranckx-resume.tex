% !TeX document-id = {29e3355d-f5f2-4d79-860a-225b1558ea09}
% !TeX spellcheck = en_GB
% !TeX program = xelatex

% LaTeX source of my resume
% =========================

% Commented for easy reuse... ;)

% See the `README.md` file for more info.

% This file is licensed under the CC-NC-ND Creative Commons license.


% Start a document with the here given default font size and paper size.
\documentclass[10pt,a4paper]{article}
\usepackage{fontspec}
\usepackage{fontawesome}
% Set the page margins.
\usepackage[a4paper,margin=0.75in]{geometry}
\usepackage{bibentry}

% Setup the language.
\usepackage[english]{babel}
\hyphenation{Some-long-word}

% Makes resume-specific commands available.
\usepackage{resume}

\usepackage{academicons}
\definecolor{orcidlogocol}{HTML}{A6CE39}

\begin{document}  % begin the content of the document
\sloppy  % this to relax whitespacing in favour of straight margins
\nobibliography{ref}
\bibliographystyle{unsrt}

% title on top of the document
\maintitle{Iwein Vranckx}{October 14, 1978}{Last update on \today}

\nobreakvspace{0.3em}  % add some page break averse vertical spacing

% \noindent prevents paragraph's first lines from indenting
% \mbox is used to obfuscate the email address
% \sbull is a spaced bullet
% \href well..
% \\ breaks the line into a new paragraph
\noindent\href{mailto:ivranckx.at.kuleuven.dot.be}{\faInbox \; iwein.vranckx\mbox{}@\mbox{}kuleuven.be}\sbull
%\noindent\href{mailto:ivranckx.at.gmail.dot.be}{ivranckx\mbox{}@\mbox{}gmail.be}\sbull
\faWhatsapp \; \textsmaller{+}32-016.43.83.29\sbull
\href{https://github.com/ivranckx}{\faGithub \; Github}\sbull
\href{https://www.researchgate.net/profile/Iwein_Vranckx}{\faPaperclip \; Researchgate}\sbull
\href{http://linkedin.com/in/ivranckx}{\faLinkedin \; Linkedin}
\\
\href{https://www.google.be/maps/place/Oudenbos+81a,+3202+Aarschot/@50.965459,4.8938466,13z/data=!4m5!3m4!1s0x47c143b94205376f:0xa562b96ef1bc6568!8m2!3d50.9657156!4d4.9015994}{\faMapMarker \; Oudenbos  81, 
3202\thinspace Rillaar, Belgium}\sbull%{\large \sc ld}\sbull
\href{https://orcid.org/0000-0002-2439-6906}{\textcolor{orcidlogocol}{\aiOrcid} \hspace{0.2mm} 0000-0002-2439-6906}

\spacedhrule{0.9em}{-0.4em}  % a horizontal line with some vertical spacing before and after

\roottitle{Summary}  % a root section title

\vspace{-1.3em}  % some vertical spacing

\begin{multicols}{2}  % open a multicolumn environment
\noindent \emph{Entrepeneurial geek with roots in the open source movement. Passionately enabling software-related teams to deliver. Creates/\,structures/\,improves processes. Architects and helps to implement future-proof solutions. Coaches both individuals and groups.}
%\\
%At the age of seven (1989) Cies wrote his first lines of code in a \acr{LOGO}-like language on an \acr{MSX} (pre-\acr{PC}).  Two years later he attended a conference on an emerging new technology, the Internet, at the Erasmus University from which he would graduate 16 years later (2000) with a degree in Business Computer Science.
%After being introduced to the open source movement in 1997, he taught himself a variety of skills including system administration and programming (Bash, Python, Ruby \& \CPP).  By 2002 he got his pet project \acr{KT}urtle ---a zero-entry-barrier programming environment--- included in \acr{KDE}'s \emph{edu} module, and thereby in almost every Linux distribution.
%Extensively travelled Europe and Asia after graduating. On return his professional life became a mix of hands-on work in startups and web software agencies, while settling as a husband and father in personal life.
\end{multicols}


\spacedhrule{0em}{-0.4em}

\roottitle{Experience}

\headedsection
  {\href{https://www.tomra.com/en/}{\acr{TOMRA Sorting NV}}}
  {\textsc{Haasrode, Belgium}} {%
  \headedsubsection
  {Senior Research Scientist - PhD Student (Baekeland mandaat)}
  {Oct \apo16 -- present}
  {\bodytext{
	Robust statistics to provide statistical methods which are not unduly affected
	by outliers: multivariate observations that deviate from normal samples. Classical chemometrical estimators (e.g. SVM's, PLS, PCA, ...) often have poor performance on datasets with outliers. Although it is fairly
	easy to show the sometimes dramatic effects caused by just
	one outlier in the majority of statistical machine learning
	techniques, the use of robust statistics in industrial machine
	learning tasks is only gradually being recognized in the last
	couple of years.  Modern industrial food sorting machines are processing several
	gigabytes of high dimensional chemometrical measurements
	in milliseconds, pushing against the boundaries of the
	available computational power. Furthermore, the amount of
	data is expected to grow over the coming years.
	The cross-fertilization of industrial machine classifiers and
	robust statistical methods in a big data scene has undoubtedly
	proven to be beneficial. However: the high computational
	demand forces the continuously exploration of new massive
	parallelization technologies and – more importantly - the
	research and development of new, robust sparse statistical
	methods which can handle modern datasets under state of
	the art, industrial real-time constraints.\\
	Programming languages: Python, MATLAB, C++, Julia
}}

  \vspace{+1em}  
  \headedsubsection
    {Senior Research Scientist}
    {Oct \apo16 -- present}
    {\bodytext{GPGPU gebaseerde classificatoren. In essentie omvat zo een efficiëntie gedreven classifi-cator een C++, GPU raamwerk waarin onderstaande technieken toegepast worden:(a) Detectie van multivariate outliers.(b) Normalisatie  klasse balancering – indien opportuun.(c) Ongesuperviseerde, statistisch aangedreven parameter optimalisatie.(d) Aansturen van het geïntegreerd crossvalidatiemodel.(e) Een Hammingsafstand gebaseerde foutcorrectiemechanisme. \\
	Programming languages: MATLAB, C++, cuBLAS, CUDA 
}}

 \vspace{+1em}  
  \headedsubsection
{Research Scientist, digital signal processing}
{Oct \apo16 -- present}
{\bodytext{Chemometrischeanalyse,onderzoek en implementatievanchemometrischemodellen,pre-processing.Mijn initieel werk was het onderzoek naar de werking en haalbaarheid van chemometri-sche methoden (PCR, O-PLS, PLS-DA, ... ) en pre processingsmethoden die goede resulta-ten gaf op onze 256-dimensionele spectrale data (er bestaan immers geen specifieke stel-regels die onmiddellijk leiden tot het vinden van de beste methode). Prototypes werdendoor mij geïmplementeerd in Java. Daarnaast had ik een groot aandeel in de inverse en-gineering van bestaande classificatoren.2. Research, Development en implementatie van het Smart Sort framework.Smart Sortis de commerciële naam van de wiskundige oplossingen die gebruikt wordenom sorteermachines volledig automatisch in te stellen. Na het aanbieden van een trai-ningsdata zorgt dit raamwerk automatisch voor de meest efficiënte machine-instellingen.Zonder in detail te gaan omvat deze technologie:(a) Verschillende entropie, Hellinger - en mutual information gebaseerde contrastbepa-lingsfilters.(b) Verschillende outlier eliminatie principes om zuivere trainingsdata te krijgen.(c) Statistische efficiëntie gebaseerde (one, multiclass) classificatoren die specifieke nor-malisatiealgoritmes gebruiken.Deze technologie is integraal geïmplementeerd geworden in Java en communiceert recht-streeks met de machinehardware over een generieke communicatielaag.\\
	Programming languages: JAVA, C++, Octave, batch}}
}

\headedsection
{\href{https://www.tomra.com/en/}{\acr{Unizo HQ}}}
{\textsc{Brussels, Belgium}} {%
	\headedsubsection
	{ICT-Consultant}
	{2010 -- 2010}
	{\bodytext{Als IT adviseur stond ik in voor de software en hardwarematige ICT-advisering aanKMO’s op basis van input van externe ICT-partners, eigen opzoekingwerk en eigen ervaringen.Ik initieerde, begeleidde en bracht ICT-veranderingstrajecten (ERP, CRM, CMS, ..) tot uitvoeringbij KMO’s en ruimer binnen een sector. Naast de bovengenoemde punten verzorgde ik de dage-lijkse communicatie en opvolging met de betrokken beroepssectoren - evenals de organisatievan zowel praktische als inhoudelijke opleidingen, workshops en netwerkevents}}}




%\headedsection
%  {\href{http://www.hoppinger.com}{Hoppinger}}
%  {\textsc{Rotterdam, The Netherlands}} {%
%  \headedsubsection
%    {Head of Technology}
%    {Apr \apo12 -- Oct \apo16}
%    {\bodytext{Hoppinger is an open source minded \emph{full-service} internet agency.  In charge of the dev't capacity, reporting directly to managing director and co-founder Marijn Bom.  Drafted and carried out the technical company vision.  Inspired both devs and management to look beyond \acr{PHP}. Grew custom app dev't into a serious revenue segment.  Implemented continuous delivery. Created and taught several in-house courses.  Enacted testing strategy.  Recruited viciously, in part by organizing educational events.  Restructured to fixed team configurations.  Introduced DevOps (Puppet).  Helped to set up the \textit{Customer Support} and \textit{IT ops} dep'ts.  Contributed to the sales pitches for, and in charge of the software architecture for, all technically challenging projects. Involved with many company-wide process improvements.}}
%}
%
%\headedsection
%  {\href{http://www.hr.nl}{Rotterdam University of Applied Sciences}}
%  {\textsc{Rotterdam, The Netherlands}} {%
%  \headedsubsection
%    {Member of the Professional Advisory Board \textnormal{\textit{(till Sep \apo16)}} \& Guest Lecturer}
%    {Sep \apo12 -- present}
%    {%\bodytext{Introductory lecture on history of software development and open source for 1\textsuperscript{st} year CS students.}
%    }
%}
%
%\headedsection  % sets the header for the section and includes any subsections
%  {\href{http://www.intellecap.com}{Intellecap}/\href{http://istpl.in}{\acr{ISTPL}}}
%  {\textsc{Mumbai, Pune \& Hyderabad, India}} {%
%
%  \headedsubsection  % sets the header for a subsection and contains usually body text
%    {\acr{IT} Consultant}
%    {Nov \apo08 -- Feb \apo09}
%    {\bodytext{Intellecap is a financial advisory firm serving corporates, non-profits and governments working in developing markets. Assessed their software development team and processes. Trained their devs. Architected and built several web apps; one being Mostfit an open source \acr{MIS} for \href{http://en.wikipedia.org/wiki/Microcredit}{microcredit} lenders.}}
%
%  \headedsubsection
%    {\acr{IT} \& Strategy Consultant}
%    {Jan \apo10 -- Aug \apo11}
%    {\bodytext{Called in to solve several technical challenges and develop growth strategies regarding Mostfit.}}
%
%  \headedsubsection
%    {CTO}
%    {Oct \apo11 -- Feb \apo12}
%    {\bodytext{Proudly joined the \acr{C}-family of Intellicap's software division, ISTPL, to make \href{https://github.com/Mostfit/mostfit}{Mostfit} the nr.1 software solution for micro credit lenders around the globe.  Shut down in six months due to lack of investment.}}
%}
%
%\headedsection
%  {\href{http://www.zarafa.com}{Zarafa}}
%  {\textsc{Delft, The Netherlands}} {%
%  \headedsubsection
%    {\acr{QA} \& Release Manager}
%    {Dec \apo09 -- Jan \apo11}
%    {\bodytext{Before 2015, when Zarafa entered in an extensive parnership with Amazon, it was probably the fastest growing open source product company in Europe, making a drop-in replacement for MS Exchange.  Reported directly to the \acr{CEO}, Brian Josef, and worked closely with the \acr{CTO}, Steve Hardy.  In charge of the 6 men strong QA dep't.  Established test automation and continuous integration.  Architected and implemented an all-integrated documentation and translation system that employed community effort.  Got sent to India to analyse and streamline their outsourced operations.}}
%}
%
%\headedsection
%  {\href{http://www.dharmapublishing.com}{Dharma Publishing}}
%  {\textsc{near San Francisco, \acr{USA}}} {%
%  \headedsubsection
%    {\acr{IT} Consultant}
%    {Nov \apo09 -- Dec \apo09}
%    {\bodytext{Dharma Publishing, the worlds largest Buddhist publisher, is a non-profit, all-volunteer organisation to preserve Tibetan buddhism and culture. Built their web shop, improved related processes.}}
%}
%
%\headedsection
%  {\href{http://www.kde.org}{KDE}}
%  {\href{http://edu.kde.org/kturtle}{edu.kde.org/kturtle}} {%
%  \headedsubsection
%    {Software Engineer}
%    {Dec \apo03 -- present}
%    {\bodytext{\acr{KT}urtle is an educational programming environment that simplifies learning the basics of programming.  \acr{KT}urtle is intended as a gift to future generations:\ a simple environment to get started with programming.  In 2003 \acr{KT}urtle got admitted to the \acr{KDE} project.}}
%}
%
%\headedsection
%  {\href{http://truetopiaproject.org}{Truetopia Project}}
%  {\href{http://truetopiaproject.org}{truetopiaproject.org}} {%
%  \headedsubsection
%    {Initiator \& Lead Developer}
%    {Nov \apo07 -- Apr \apo10}
%    {\bodytext{The Truetopia Project is an open source web application (Rails) to facilitate self-governing communities.  It provides a workflow for collaborative problem identification and solution design.}}
%}
%
%\headedsection
%  {\href{http://www.dpu.ac.th/dpuic}{Dhurakij Pundit University International College}}
%  {\textsc{Bangkok, Thailand}} {%
%  \headedsubsection
%    {Guest Lecturer}
%    {Sep \apo09}
%    {\bodytext{Invited by Dr.\@ Pilun Piyasirivej and Mr.\@ Michel Bauwens to deliver several guest lectures.}}
%}
%
%\headedsection
%  {\href{http://www.opendream.th}{Opendream}}
%  {\textsc{Bangkok, Thailand}} {%
%  \headedsubsection
%    {Freelance \acr{IT} Consultant}
%    {Aug \apo09 -- Sep \apo09}
%    {\bodytext{Architected and largely implemented an open source media sharing web service (\acr{REST API}) that facilitates video uploads, transcoding and streaming.  Coached their development team on system design, Ruby web development and software engineering methodologies (Scrum/\acr{TDD}/\acr{BDD}).}}
%}
%
%\headedsection
%  {\href{http://www.commuun.nl}{Commuun}}
%  {\textsc{Rotterdam, The Netherlands}} {%
%  \headedsubsection
%    {Senior Visionary}
%    {Jul \apo06 -- Sep \apo09}
%    {\bodytext{Set up the technical infrastructure, defined the business cases, created the brand and built several Rails-based web applications with Peter Duijnstee, my close friend and proprietor of Commuun.}}
%}
%
%\headedsection
%  {\href{http://www.eur.nl}{Erasmus University Rotterdam}}
%  {\textsc{Rotterdam, The Netherlands}} {%
%  \headedsubsection
%    {Guest Lecturer}
%    {Jul \apo06 -- Jul \apo09}
%    {\bodytext{Spoke on the phenomenon of open source on request of Prof.\ Michiel van Wezel.}}
%}
%
%%\headedsection
%%  {LIP Automatisering}
%%  {\textsc{Breda, The Netherlands}} {%
%%  \headedsubsection
%%    {Software Auditor}
%%    {Sep \apo06}
%%    {\bodytext{Audited their flag ship product \emph{\acr{LIP} Suite}:\ an %\acr{ERP} solution for construction companies.}}
%%}
%
%\headedsection
%  {
%    \href{http://www.thehealthagency.com}{The Health Agency}
%    \acr{(now \href{http://www.minddistrict.com}{MindDistrict})}
%  }
%  {\textsc{Delft \& Rotterdam, The Netherlands}} {%
%
%  \headedsubsection
%    {Software Engineer}
%    {Jun \apo05 -- Feb \apo06}
%    {\bodytext{Worked on their \acr{CMS} (written in Python and uses Postgre\acr{SQL}, \acr{XML}/\acr{XSLT} and Twisted).}}
%
%  \headedsubsection
%    {Software Auditor}
%    {Dec \apo06}
%    {\bodytext{Assessed their Python/Zope/\acr{Z}o\acr{DB}-based product re-engineering effort.}}
%}

\vspace{-0.2em}
\begin{center}
  \emph{\small Please refer to \href{http://www.linkedin.com/in/ivranckx}{my LinkedIn profile} for a more complete list of work experience.}
\end{center}


\spacedhrule{-0.2em}{-0.4em}

\roottitle{Education}

\headedsection
  {\href{https://wet.kuleuven.be/english}{KU Leuven, faculty of Science}}
  {\textsc{Leuven, Belgium}} {%
  \headedsubsection
    {Doctor of Science (PhD): Statistics (dr.)}
    {2016 -- 2020}{\\Development of real-time, robust statistical methods with novel applications in food sorting}
}
\vspace{+0.2em}
\headedsection
  {\href{https://eng.kuleuven.be/en}{KU Leuven, faculty of Engineering Science}}
  {\textsc{Leuven, Belgium}} {%
  \headedsubsection
    {Master of Electrical Engineering: Information Systems and signal Processing (ir.)}
    {2005--2009} {\\Development of an SVM-based OCS for Latin-Greek manuscripts}
}

\vspace{+0.2em}
\headedsection
  {\href{https://www.uantwerpen.be/en/}{University of Antwerp, faculty of Biomedical Sciences}}
  {\textsc{Antwerp, Belgium}} {%
  \headedsubsection
    {Master of Biomedical Sciences: Neurosciences (Research)}
    {2004 -- 2005} {\\Reduction of ring artefacts on \acr{$\mu$-CT} images}
}

\vspace{+0.2em}
\headedsection
{\href{https://www.kdg.be/}{Karel de Grote-Hogeschool}}
{\textsc{Antwerp, Belgium}} {%
	\headedsubsection
	{Master in Electronics and ICT Engineering Technology (ing.)}
	{2001 -- 2004} {\\Tristan: data acquisition software for a heat treatment production process}
}

\vspace{+0.2em}
\headedsection
{\href{https://www.kdg.be/}{Karel de Grote-Hogeschool}}
{\textsc{Antwerp, Belgium}} {%
	\headedsubsection
	{Bachelor of Electromechanics}
	{1998 -- 2001} {\\Development of a (SQL-based) database application for the registration of production data}
}

\spacedhrule{0.5em}{-0.4em}

\roottitle{Publications}

\inlineheadsection  % special section that has an inline header with a 'hanging' paragraph
{Published:}
{
	I.~Vranckx. Development of real-time, robust statistical methods with novel applications in food sorting (2020). Dissertation-thesis.
	
	\hangindent=3.5mm
	\hangafter=1		
	I. Vranckx, J. Raymaekers, B. De Ketelaere, P. J. Rousseeuw, M. Hubert (2021).
	\newblock Real-time discriminant analysis in the presence of label and measurement noise.
	\newblock {\em Chemometrics and Intelligent Laboratory Systems\/}~{\em 208}.
	
	De~Ketelaere, B., M.~Hubert, J.~Raymaekers, P.~J. Rousseeuw, and I.~Vranckx	(2020).
	\newblock Real-time outlier detection for large datasets by {RT-DetMCD}.
	\newblock {\em Chemometrics and Intelligent Laboratory Systems\/}~{\em 199}. 
		
	Raymaekers, J., P.~J. Rousseeuw, and I.~Vranckx (2018).
	\newblock Discussion of ``{T}he power of monitoring: how to make the most of a contaminated multivariate sample''.
	\newblock {\em Statistical Methods \& Applications\/}~{\em 27}, 589--594. 
}
\vspace{0.5em}
\inlineheadsection
{Under review:}
{J. Schreurs, I. Vranckx, B. De Ketelaere, M. Hubert, J. A.K. Suykens, P. J. Rousseeuw (2020). Outlier detection in non-elliptical data by kernel MRCD. \textit{ArXiv e-prints}, arXiv:2008.02046. I. Vranckx and J. Schreurs contributed equally to this work.}

\spacedhrule{1.6em}{-0.4em}

\roottitle{Skills}

\inlineheadsection  % special section that has an inline header with a 'hanging' paragraph
{Technical expertise:}
{Leading and recruiting teams of software engineers.  Big fan of Agile methodologies, continuous delivery and functional programming.  Enjoys writing Ruby/\nsp Python/\nsp Java/\nsp \CPP~and Haskell.  Solid knowledge of the full web technology stack.  Able to architect \textit{and} implement distributed/\acr{HA} systems.  Strong Linux administration skills (e.g.\ Bash scripting, Apache/\acr{NGINX}, Postgres/My/No\acr{SQL}, ElasticSearch).  Well experienced with virtualization/containerization (Docker/Kubernetes, \acr{KVM}, Xen and several \acr{AWS} solutions) and DevOps (Puppet).  Sublime text, Gitkraken and Pycharm user.}

\vspace{0.5em}
\inlineheadsection
{Natural languages:}
{Dutch \emph{(mother tongue)}, English \emph{(professional proficiency)}, French \emph{(elementary proficiency)}}

\spacedhrule{1.6em}{-0.4em}

\roottitle{Interests}

\inlineheadsection	
{Research interests (non-exhaustive):}
{Data mining, (robust) calibration methods, (non-linear) classification, robust statistics, big data, CNN's, real-time GP-GPU \& MapReduce based processing, SVM's.}

\inlineheadsection	
  {General interests (non-exhaustive)):}
  {art, Buddhism, cryptography, functional programming, Go (board game), history, music (from classical and jazz to Berlin-techno), \acr{NLP}, permaculture, philosophy, rock climbing, startups, travel, typography (e.g.\ graphic design, \LaTeX), \acr{UX}-design and vegan cuisine.}


\end{document}

